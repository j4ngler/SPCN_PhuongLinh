\documentclass[12pt,a4paper]{article}
\usepackage[utf8]{inputenc}
\usepackage[vietnamese]{babel}
\usepackage{enumitem}
\usepackage{hyperref}
\usepackage{geometry}
\geometry{margin=2.5cm}

\title{Giáo án Buổi 4 (thực hành) \\ \small Nền tảng Học tập Cá nhân hóa}
\author{}
\date{}

\begin{document}
\maketitle

\section*{Mục tiêu (buổi thực hành)}
\begin{itemize}[leftmargin=1.2cm]
    \item Tập trung thực hành end-to-end, giảm lý thuyết (buổi 3 đã nhiều lý thuyết).
    \item Vận hành pipeline: nạp dữ liệu, làm sạch, sinh đặc trưng, dự đoán AI\_Score, sinh gợi ý.
    \item Rèn kỹ năng xử lý lỗi thực tế: CSV sai cột, thiếu year/semester, lỗi pandas, lỗi gợi ý.
    \item Kiểm tra và chỉnh sửa giao diện: nhập dữ liệu mới, chạy gợi ý, đọc log và sửa lỗi hiển thị.
\end{itemize}

\section*{Nội dung chi tiết (120 phút, thiên về thực hành)}
\subsection*{1) Khởi động ngắn (5')}
\begin{itemize}[leftmargin=1.2cm]
    \item Nhắc nhanh pipeline và checklist kiểm thử (input $\rightarrow$ clean $\rightarrow$ features $\rightarrow$ predict $\rightarrow$ gợi ý).
    \item Chia nhóm, giao bộ CSV mẫu và nhiệm vụ thực hành.
\end{itemize}

\subsection*{2) Lý thuyết tối giản (10')}
\begin{itemize}[leftmargin=1.2cm]
    \item Nhắc tiêu chí AI\_Score (clip 0--1, dùng học kỳ mới nhất) và gợi ý ưu tiên môn điểm thấp.
    \item Quy tắc dữ liệu sạch: cột bắt buộc, mã môn thống nhất, year/semester không trống.
\end{itemize}

\subsection*{3) Demo End-to-End nhanh (15')}
\begin{itemize}[leftmargin=1.2cm]
    \item Kiểm tra input (CSV, encoding, delimiter, cột bắt buộc).
    \item Chạy pipeline: xử lý dữ liệu $\rightarrow$ features $\rightarrow$ predict AI\_Score $\rightarrow$ tạo gợi ý.
    \item Mở dashboard, nhấn ``Tạo gợi ý'', kiểm tra file output (\texttt{ai\_scores.csv}, \texttt{recommendations.csv}) và giao diện.
\end{itemize}

\subsection*{4) Thực hành chính (60')}
\begin{itemize}[leftmargin=1.2cm]
    \item Bài 1: Thêm 1 học sinh mới (profile + điểm + feedback) vào CSV; chạy pipeline; xem AI\_Score + gợi ý trên dashboard.
    \item Bài 2: Cố tình làm sai CSV (thiếu/thừa cột hoặc sai mã môn) $\rightarrow$ chạy pipeline $\rightarrow$ đọc traceback/log $\rightarrow$ sửa đúng.
    \item Bài 3: Điều chỉnh điểm một môn để AI\_Score tăng/giảm, chạy lại và quan sát gợi ý thay đổi.
    \item Bài 4: Bỏ \texttt{year/semester} để thấy dữ liệu bị lọc hết; thêm lại để phục hồi.
    \item Bài 5 (nhanh): Kiểm tra \texttt{career\_path} nếu có — thêm 1 giá trị hợp lệ, chạy lại, xem lý do gợi ý có đổi không.
\end{itemize}

\subsection*{5) Phân tích \& Thảo luận (15')}
\begin{itemize}[leftmargin=1.2cm]
    \item Khi nào nên ưu tiên môn AI\_Score thấp? Khi nào giữ môn trung bình để duy trì?
    \item Cách điền mặc định nếu thiếu year/semester; rủi ro lọc sai kỳ.
    \item Gợi ý cải tiến: trọng số môn chuyên ngành, ngưỡng động theo từng học sinh, giải thích gợi ý rõ hơn.
\end{itemize}

\subsection*{6) Tổng kết \& Bài tập về nhà (15')}
\begin{itemize}[leftmargin=1.2cm]
    \item Checklist: dữ liệu sạch, pipeline chạy, AI\_Score sinh ra, gợi ý hiển thị, không lỗi pandas/CSV, dashboard ổn.
    \item Bài tập: nhập thêm 3--5 học sinh, chạy pipeline, chụp màn hình dashboard + gợi ý, ghi lại lỗi (nếu có) và cách xử lý.
    \item Chuẩn bị buổi sau: thử điều chỉnh feature/mô hình hoặc bổ sung logic gợi ý theo \texttt{career\_path}.
\end{itemize}

\section*{Tài liệu \& Chuẩn bị}
\begin{itemize}[leftmargin=1.2cm]
    \item Máy đã cài môi trường dự án (Python, venv, pandas, Flask), có thể chạy pipeline và web.
    \item Bộ CSV mẫu (input) và file output mẫu để đối chiếu.
    \item Slide ngắn: AI\_Score, pipeline, lỗi thường gặp; sơ đồ luồng dữ liệu.
    \item Checklist thao tác: nhập dữ liệu $\rightarrow$ chạy pipeline $\rightarrow$ kiểm tra output $\rightarrow$ mở web $\rightarrow$ tạo gợi ý $\rightarrow$ xem log khi lỗi.
\end{itemize}

\end{document}

